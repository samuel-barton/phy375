\section*{Problem 1}

In this problem we are asked to find the relation between $\delta$ and the angles in the provided triangle. We first note that this
triangle is isocoles and thus the two bottom angles are equal, by extension of the definition of isocoles triangles. Next we note that 
the angle B as labeled in the given triangle must be $\pi - \mathrm{A}$ since the sum of the four angles of any quadrilateral must be 
$360^\circ$ and the other two angles are formed by lines normal to their edges of the triangle and thus are $90^\circ$ angles.
\\
\\
We use Snell's Law to get the value for $\theta^{'}_1$ remembering that $n_i = 1$ and $n_t = n$

$$\sin\theta_1 = n\sin\theta^{'}_1$$ 

\begin{equation} \label{theta-1-prime}
	\theta^{'}_1 = \sin^{-1}\left(\frac{\sin\theta_1}{n}\right)
\end{equation}

We can now get the value of $\theta^{'}_2$ if we remember that the sum of the angles in a triangle must be $\pi$ radians

$$\theta^{'}_1 + \theta^{'}_2 + \mathrm{B} = \pi$$

\begin{equation*}
	\theta^{'}_2 = \pi - \mathrm{B} - \theta^{'}_1
\end{equation*}

Since $\mathrm{B} = \pi - \mathrm{A}$, we can simplify the above equation as follows

$$
	\theta^{'}_2 = \pi - \theta^{'}_1 - (\pi - \mathrm{A})
$$
$$
	\theta^{'}_2 = \pi - \theta^{'}_1 - \pi + \mathrm{A}
$$
$$
	\theta^{'}_2 = - \theta^{'}_1 + \mathrm{A}
$$
$$
	\theta^{'}_2 = \mathrm{A} - \theta^{'}_1
$$


If we now substitute in equation \ref{theta-1-prime} we get 

\begin{equation} \label{theta-2-prime}
	\theta^{'}_2 = \mathrm{A} - \sin^{-1}\left(\frac{\sin\theta_1}{n}\right)
\end{equation}

With $\theta^{'}_2$, we can now calculate $\theta_2$ by again using Snell's Law, except this time $n_i = n$ and $n_t = 1$.

$$
	n\sin\theta^{'}_2 = \sin\theta_2
$$ 
$$
	\theta_2 = \sin^{-1}\left(n\sin\theta^{'}_2\right)
$$

\begin{equation} \label{theta-2}
	\theta_2 = \sin^{-1}(n\sin\left[ \mathrm{A} - \sin^{-1}\left(\frac{\sin\theta_1}{n}\right) \right]
\end{equation}

We now calculate $\delta_1$ and $\delta_2$ by using the geometric principle that the opposite angles of any two intersecting lines are 
equal. Thus 

$$
	\theta_1 = \delta_1 + \theta^{'}_1
$$
$$
	\delta_1 = \theta_1 - \theta^{'}_1
$$

Using equation \ref{theta-1-prime} we get

\begin{equation} \label{delta-1}
	\delta_1 = \theta_1 - \sin^{-1}\left(\frac{\sin\theta_1}{n}\right)
\end{equation}

and

$$
	\theta_2 = \delta_2 + \theta^{'}_2
$$
$$
	\delta_2 = \theta_2 - \theta^{'}_2
$$

Using equations \ref{theta-2-prime} and \ref{theta-2} we get
$$
	\delta_2 = \sin^{-1}(n\sin\left[ \mathrm{A} - \sin^{-1}\left(\frac{\sin\theta_1}{n}\right) \right]
  	- \left[\mathrm{A} - \sin^{-1}\left(\frac{\sin\theta_1}{n}\right)\right]
$$
\begin{equation} \label{delta-2}
	\delta_2 = \sin^{-1}(n\sin\left[ \mathrm{A} - \sin^{-1}\left(\frac{\sin\theta_1}{n}\right) \right]
  	- \mathrm{A} + \sin^{-1}\left(\frac{\sin\theta_1}{n}\right)
\end{equation}

With values for $\delta_1$ and $\delta_2$ we again use the principle that the sum of the angles of a triangle must be $\pi$ raidans to 
get the angle C, where C is the third angle in the triangle $\delta_1\delta_2\mathrm{C}$

$$\delta_1 + \delta_2 + \mathrm{C} = \pi$$

\begin{equation} \label{C}
	\mathrm{C} = \pi - \delta_1 - \delta_2
\end{equation}

With the value for C we can finally calculate the value $\delta$ by using the geometric theorem that the sum of the interior and exterior
angles of a line relative to another line must equal $\pi$ radians. In this case the lines are that of the incident ray, and the ray which
exits the prism. The angle C is the exterior angle to the incident ray relative to the final ray, and the angle $\delta$ is the internal
angle.

\begin{equation} \label{delta}
	\delta = \pi - \mathrm{C}
\end{equation}

Now we begin back-substitution starting with equation \ref{C}.

$$
	\delta = \pi - (\pi - \delta_1 - \delta_2)
$$

Next we substitute in equations \ref{delta-1} and \ref{delta-2} to get the final expression for $\delta$

$$
	\delta = \pi - \left[\pi - \left[\theta_1 - \sin^{-1}\left(\frac{\sin\theta_1}{n}\right)\right] - 
	\left[ \sin^{-1}(n\sin\left[ \mathrm{A} - \sin^{-1}\left(\frac{\sin\theta_1}{n}\right) \right]
  	- \mathrm{A} + \sin^{-1}\left(\frac{\sin\theta_1}{n}\right) \right]\right]
$$
$$
	\delta = \pi - \left[\pi - \theta_1 + \sin^{-1}\left(\frac{\sin\theta_1}{n}\right) - 
	\sin^{-1}(n\sin\left[ \mathrm{A} - \sin^{-1}\left(\frac{\sin\theta_1}{n}\right) \right]
  	+ \mathrm{A} - \sin^{-1}\left(\frac{\sin\theta_1}{n}\right) \right]
$$
$$
	\delta = \pi - \left[\pi - \theta_1 - 
	\sin^{-1}(n\sin\left[ \mathrm{A} - \sin^{-1}\left(\frac{\sin\theta_1}{n}\right) \right]
  	+ \mathrm{A} \right]
$$
$$
	\delta = \pi - \pi + \theta_1 + 
	\sin^{-1}(n\sin\left[ \mathrm{A} - \sin^{-1}\left(\frac{\sin\theta_1}{n}\right) \right]
  	- \mathrm{A}
$$

And finally we reach the value for $\delta$
\begin{equation*} \label{delta-simplified}
	\delta = \theta_1 + 
	\sin^{-1}(n\sin\left[ \mathrm{A} - \sin^{-1}\left(\frac{\sin\theta_1}{n}\right) \right]
  	- \mathrm{A}	
\end{equation*}
This relation gives us $\delta$ as a function of $\theta_1$ and A. 


