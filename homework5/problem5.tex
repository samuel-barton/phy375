\section*{Problem 5}

We now consider why a fish, or any observer underwater, will see a cone-framed image of what is above the
surface surrounded by darkness. Let us first remember Snell's Law which states that 
$$n_i\sin\theta_i = n_t\sin\theta_t$$
Using Snell's Law we can define the Crittical angle for the transmission between two medii as the angle 
$\theta_c$ such that $\sin\theta_c = 1 \implies \theta_c = \frac{\pi}{2}$. If we use this definition in
conjunction with Snell's law we can get an expression for $\theta_c$ between any two medii

\begin{equation*}
	n_i\sin\theta_c = n_t\sin(\frac{\pi}{2}) = n_t 
	\implies \sin\theta_c = \frac{n_t}{n_i} 
	\implies \theta_c = \sin^{-1}\left(\frac{n_t}{n_i} \right)
\end{equation*}
\\
We can use this in the case of water and air, noting that $n_{\mathrm{water}} = 1.3374$ and 
$n_{\mathrm{air}} = 1.0028$, to find the crittical angle where we have total internal reflection. The reason
why this angle defines the cone seen from underwater, it's twice this angle but that's beside the point, is
that light incident on the water at any angle greater than $0^\circ$ relative to the origin will have
a transmitted portion in the water, and since there can be no light coming from the surface at less than 
$0^\circ$ the crittical angle defines the bound of the cone of light. So the $180^\circ$ hemisphere above the
water is compressed down into a cone whose angle is twice $\theta_c$. The reason why the area outside the 
cone is so dark is taht it is lit only by that light which reflects off the surface of the water from below and
comes back down, so it must be much darker as there are no light sources of great magnitude, or any amazing mirrors underwater. 
\\
\\
The angle $\theta_c$ for water to air is $\sin^{-1}\left(\frac{1.0028}{1.3374}\right) = 48.57^\circ$. This
means that the angle of the cone will be 
$$\theta_{\mathrm{cone}} = 2 \times \theta_c = 97.15^\circ$$

