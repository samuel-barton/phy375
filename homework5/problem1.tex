	\section*{Problem 1}
	
	Let us consider light travelling through air ($n_{\mathrm{air}} = 1.00$)  hitting a glass surface ($n_{\mathrm{glass}} = 1.60$) with an incident angle
	$\theta_i = 30^{\circ}$. 
	
	\subsection*{Part a}
	
	We are now to calculate the reflection and transmission coefficients for both p-polarized and s-polarized light.
	\\
	\\
	First we have the equations for the reflection coefficients
	
	\begin{equation} \label{r_par}
		\mathrm{r}_\parallel = \dfrac{\tan (\theta_i - \theta_t)}{\tan (\theta_i + \theta_t)}
	\end{equation}
	
	\begin{equation} \label{r_per}
		\mathrm{r}_\bot = -\dfrac{\sin (\theta_i - \theta_t)}{\sin (\theta_i + \theta_t)}
	\end{equation}
	\\
	Then we have the equations for the transmission coeffiients
	
	\begin{equation} \label{t_par}
		\mathrm{t}_\parallel = \dfrac{2\sin\theta_t\cos\theta_i}{\sin(\theta_i + \theta_t)}
	\end{equation}
	
	\begin{equation} \label{t_per}
		\mathrm{t}_\bot = \dfrac{2\sin\theta_t\cos\theta_i}{\sin(\theta_i + \theta_t)\cos(\theta_i - \theta_t)}
	\end{equation}
	\\
	Using these four equations we can now calculate all the values for both types of light incident on the glass. Before we continue though, we need
	to calculate the value of $\theta_t$. We can find this easily enough using Snell's Law
	\begin{equation*}
		n_i \sin \theta_i = n_t \sin \theta_t \; \implies 
		\; \sin \theta_t = \dfrac{n_i \sin\theta_i}{n_t} \; \implies 
		\; \theta_t = \sin^{-1}\Big(\dfrac{n_i \sin\theta_i}{n_t}\Big)
	\end{equation*}
	
	If we insert the known values, namely $n_i$, $n_t$, and $\theta_i$ we can calculate $\theta_t$.
	\begin{equation*}
		\theta_t = \sin^{-1}\Big(\dfrac{\sin (30^\circ)}{1.60}\Big) = 18.2^\circ
	\end{equation*}
	Now that we have a value for $\theta_t$ we can now calculate the four values for the reflection and transmission coefficients. First we'll 
	consider the two values for p-polarized light using equations \ref{r_par} and \ref{t_par}. 
	
	\begin{equation*}
		\mathrm{r}_\parallel = \dfrac{\tan (30^\circ - 18.2^\circ)}{\tan (30^\circ + 18.2^\circ)} = 
		\dfrac{\tan (11.8^\circ)}{\tan (48.2^\circ)} = \dfrac{0.209}{1.118} = 0.187
	\end{equation*}
	and
	\begin{equation*}
		\mathrm{t}_\parallel = \dfrac{2\sin(18.2^\circ)\cos(30^\circ)}{\sin(30^\circ + 18.2^\circ)} =
		\dfrac{2\sin(18.2^\circ)\cos(30^\circ)}{\sin(48.2^\circ)} = \dfrac{2 \times 0.312 \times 0.866}{0.745} = 0.725 
	\end{equation*}
	\\
	Now on to the s-polarized light where we'll use equations \ref{r_per} and \ref{t_per}
	\begin{equation*}
		\mathrm{r}_\bot = -\dfrac{\sin (30^\circ - 18.2^\circ)}{\sin (30^\circ + 18.2^\circ)} = 
		-\dfrac{\sin (11.8^\circ)}{\sin (48.2^\circ)} = -\dfrac{0.204}{0.745} = -0.274 
	\end{equation*}
	and
	\begin{equation*}
		\mathrm{t}_\bot = \dfrac{2\sin18.2^\circ\cos30^\circ}{\sin(30^\circ + 18.2^\circ)\cos(30^\circ - 18.2^\circ)} = 
		\dfrac{2\sin18.2^\circ\cos30^\circ}{\sin(48.2^\circ)\cos(11.8^\circ)} = 
		\dfrac{2 \times 0.312 \times 0.866}{0.745 \times 0.979} = 0.741
	\end{equation*}
	
	This gives us the following four values
	$$ \mathrm{r}_\parallel = 0.187 $$
	$$ \mathrm{t}_\parallel = 0.725 $$
	$$ \mathrm{r}_\bot = -0.274 $$
	$$ \mathrm{t}_\bot = 0.741 $$
